\documentclass[11pt]{article}
\usepackage[utf8]{inputenc}

\usepackage{bm}
\usepackage{amsmath,amsthm,amsfonts,amssymb,amscd}
\usepackage{mathtools}

\usepackage{fullpage}

\usepackage{enumitem}

\usepackage{mathrsfs}
\usepackage{wrapfig}
\usepackage{setspace}
\usepackage{calc}
\usepackage{multicol}
\usepackage{cancel}

\usepackage{amsmath}

\begin{document}
\setcounter{section}{0}
\thispagestyle{empty}

\begin{center}
{\LARGE \bf Note}\\
\end{center}
A curve is represented by a finite set of points of cardinal $N$ such that:

\begin{itemize}
\item The distance between two consecutive points is controlled:
\[A_1 : x \to (x[2]-x[1], \cdots, x[1]-x[n] \]
(circular curve, with condition of $x[1]-x[n]$)

\item The acceleration
\[A_2: x \to (x[2] - x[1], \cdots, x[n] - x[n-1])\]
(non-circular)
\end{itemize}

$A_1$ can be represented by the matrix:
\[\begin{bmatrix}
    -1 & 1 & 0 & \cdots & 0 \\
    0 & -1 & 1 & 0 & \cdots & 0 \\
    \vdots & & & & & \\
    1 & \cdots & & & & -1
  \end{bmatrix}\]

To control the distance, we rely on two different types of
constraints: kinematic constraint and geometrical constraint.

\paragraph{Kinematic constraints.} We choose two constants $c_1$, $c_2$ to
bound the norm of each entry of:

% \begin{align*}
%   \| (A_1x)[i] \| & \leq c_1 \\
%   \| (A^T \cdot A x)[i] \| & \leq c_2
% \end{align*}

Then the base space we want to optimize on is 

\[ X = \{ x \in \Omega^n, \cdots  \} \]

which is the space of the discrete curve with distance bounded by
kinematic constraints.

Convex domain.

\paragraph{Geometrical constraints.} These constraints are simply the
intrinsic property of the curve: length and curvature (basic notions
of differential geometry, curvature represents how curvy the curve is)

We take an arc-length parametrization of the curve, i.e., a curve is
represented by a continuous function $s(t) : [0,T] \to \Omega$ with $|
s^{\prime}(t) | = 1$. So the length of the curve is $T$ and the
curvature is a function: $[0,T] \to R$ defined as
\[t \to | s^{\prime \prime}(t) |.\]

% \begin{align*}
%   \| (A_1 x)[i] \| & = c_1 \\

%   \| (A_2 x)[i] \| & \leq c_2
% \end{align*}

Non-convex space.

Linear constraints: not important for now

\section{Algorithm}
Choose $N$, we optimize with $N$ points but with
constraints, i.e., a discrete subspace $X$ of $\Omega^N$.

That means after computing new value in gradient descent, we need a projection.

Euclidean projector: $\Pi_{X} : \Omega^n \to X$

\[\Pi_{X}(z) = {\rm argmin}_{x \in X} \frac{1}{2} \| x - z \|^2 \]

\[ = {\rm argmin}_{Ax \in Y} \frac{1}{2} \| x - z \|^2 \]

When $X$ is convex, obviously the projection of $z$ is unique.

How to compute the projection?

$Y_1$, $Y_2$ the constraint set:
\[Y_1 = \{ y \in \mathbb{R}^{n \times 2} \; | \; |y_i| \leq c_1 \forall i \}\]


\[(\beta A^T \cdot A + I) \cdot x \]
\end{document}